\section{Related Work}
Numerous image processing and machine learning attempts have been made to tackle the invoice recognition problem from different angles. 

Image processing approaches rely on column detection and word sequence recognition within each logically segmented region \cite{marinai2008introduction}, with the occasional aid of machine learning techniques for more precise region classification results. While image processing techniques \cite{zhou2000hough} can be of great aid to many other models, a recognition algorithm centered around it overly simplifies the complex nature of invoice layout, and assumes homogeneous properties in region segmentation based on linear combination of rules, which is almost never the case in practice given the unpredictable nature of invoice layout.

In need of more complex models leveraging machine learning techniques, template based classification algorithms are proposed, where the template of an invoice (calculated and represented by a set of layout attributes) is either matched against a template library \cite{ming2003research}\cite{sorio2013machine}, or is assigned to a cluster of templates sharing similar properties \cite{hamza2008incremental}. In either approach, the template library or cluster constantly expands when no obvious match exists.

Template based models have the obvious benefit of being able to recognize the entire invoice all at once through pre-established template-specific rules, and perform the best with high quality images and highly distinctive templates. Unfortunately, neither is guaranteed in reality as invoices are often poorly scanned, and the huge vendor (and thus invoice template) pool implies frequent occurrence of invoices with similar structure but minor (yet crucial) variance in layout and field arrangements. In the case of template library, this could cause critical fields to be mis-recognized following incorrect rules. In the case of clusters, defining template "distance" and distinguishing minor variance within a cluster are themselves tricky and error-prone. Such models are also memory-intensive as the library or cluster size constantly expands.

Also available are rule-based models, where sets of hand-crafted rules are weighted to capture micro-level details for each field \cite{belaid2004morphological}. Such approaches avoid the inflexibility when an invoice is treated as a whole. While hand-crafted rules work well with invoices containing industry standard components and layout, they imply presumptions on field properties that are either incorrectly arbitrary (e.g., amounts are not always right aligned) or simply not achievable (e.g., match price and quantity with amount to identify invoice lines while not all three fields are present in many real-world invoices).

The model proposed in this project also performs field-by-field recognition, though instead of merely determining the weights of hand-crafted rules, a bag of potential features is supplied to generate the best set of rules that should be used.
